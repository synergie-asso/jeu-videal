\section{Exercices}






\subsection{Positif ou négatif}


\paragraph{}
Cette fonction test si le nombre passé en paramètre est positif ou négatif.
Elle retourne vrai ('True') si il est positif, et faux ('False') si il est négatif.
\begin{pythonCode}
def positif(x):
\end{pythonCode}


\subsection{Pair ou impair}


\paragraph{}
Cette fonction test si le nombre passé en paramètre est pair ou impair.
Elle retourne vrai ('True') si il est pair, et faux ('False') si il est impair.
\begin{pythonCode}
def pair(n):
\end{pythonCode}



\subsection{Egale 10}


\paragraph{}
Cette fonction test si la somme des 2 nombres passés en paramètre est égale à 10.
Elle retourne vrai ('True') si c'est le cas, et faux ('False') si ce n'est pas le cas.
\begin{pythonCode}
def egalDix(x, y):
\end{pythonCode}




\subsection{Somme d'un tableau égal à 10}
Cette fonction test si la somme des nombres d'un tableau passés en paramètre est égale à 10.
Le tableau "tab", contient des nombres, si la somme de tous ces nombre est égale à 10, la fonction retourne vrai, sinon elle retourne faux.
\begin{pythonCode}
def tableauEgalDix(tab):
\end{pythonCode}





